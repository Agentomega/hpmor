
Chapter 35: Coordination Problems, Pt 3\\
</div><div  class='storycontent nocopy' id='storycontent' ><p>They had gone to 
the Defense Professor's office, and Professor Quirrell had sealed the door 
before he leaned back in his chair and spoke.

The Defense Professor's voice was very calm, and that unnerved Harry a good 
deal more than if Professor Quirrell had been shouting.

"I am trying," said Professor Quirrell quietly, "to make allowances for the 
fact that you are young. That I myself, at the same age, was a quite 
extraordinary fool. You speak with adult style and meddle in adult games, and 
sometimes I forget that you are only a meddler. I hope, Mr.~Potter, that your 
childish meddling has not just killed you, ruined your country, and lost the 
next war."

It was very hard for Harry to control his breathing. "Professor Quirrell, I 
said a good deal less than I wished to say, but I had to say something. Your 
proposals are extremely alarming to anyone who has the slightest familiarity 
with Muggle history over the last century. The Italian fascists, some very 
nasty people, got their name from the \emph{fasces,} a bundle of rods bound 
together to symbolize the idea that unity is strength---"

"So the nasty Italian fascists believed that unity is stronger than division," 
said Professor Quirrell. Sharpness was beginning to creep into his voice. 
"Perhaps they also believed that the sky is blue, and advocated a policy of not 
dropping rocks on your head."

\emph{Reversed stupidity is not intelligence; the world's stupidest person may 
say the sun is shining, but that doesn't make it dark out{\ldots}} "Fine, 
you're right, that was an ad hominem argument, it's not wrong \emph{because} 
the fascists said it. But Professor Quirrell, you can't have everyone in a 
country take the Mark of one dictator! It's a single point of failure! Look, 
I'll put it this way. Suppose the enemy just Imperiuses whoever controls the 
Mark---"

"Powerful wizards are not so easy to Imperius," said Professor Quirrell dryly. 
"And if you cannot find a worthy leader, you are in any case doomed. But worthy 
leaders do exist; the question is whether the people shall follow them."

Harry raked his hands through his hair in frustration. He wanted to call a 
time-out and make Professor Quirrell read \emph{The Rise and Fall of the Third 
Reich} and then start the conversation over again. "I don't suppose that if I 
suggested democracy was a better form of government than dictatorship---"

"I see," said Professor Quirrell. His eyes closed briefly, then opened. 
"Mr.~Potter, the stupidity of Quidditch is transparent to you because you did 
not grow up revering the game. If you had never heard of elections, Mr.~Potter, 
and you simply \emph{saw what is there}, what you saw would not please you. 
Look to our elected Minister of Magic. Is he the wisest, the strongest, the 
greatest of our nation? No; he is a buffoon who is owned in fee simple by 
Lucius Malfoy. Wizards went to the polls and chose between Cornelius Fudge and 
Tania Leach, who had competed with each other in a grand and entertaining 
contest after the \emph{Daily Prophet,} which Lucius Malfoy also controls, 
decided that they were the only serious candidates. That Cornelius Fudge was 
genuinely selected as the best leader our country could offer is not a 
suggestion anyone could make with a straight face. It is no different in the 
Muggle world, from what I have heard and seen; the last Muggle newspaper I read 
mentioned that the previous President of the United States had been a retired 
movie actor. If you had not grown up with elections, Mr.~Potter, they would be 
as transparently silly to you as Quidditch."

Harry sat there with his mouth open, struggling for words. "The point of 
elections isn't to produce the one best leader, it's to keep politicians scared 
enough of the voters that they don't go completely evil like dictators do---"

"The last war, Mr.~Potter, was fought between the Dark Lord and Dumbledore. And 
while Dumbledore was a flawed leader who was losing the war, it is 
\emph{ridiculous} to suggest that \emph{any} of the Ministers of Magic elected 
during that period could have taken Dumbledore's place! Strength flows from 
powerful wizards and their followers, not from elections and the fools they 
elect. That is the lesson of magical Britain's recent history; and I doubt that 
the next war will teach you a lesson any different. \emph{If} you survive it, 
Mr.~Potter, which you will \emph{not} do unless you abandon the enthusiastic 
illusions of childhood!"

"If you think there are no dangers in the course of action you advocate," said 
Harry, and despite everything his voice was growing sharp, "then that, too, is 
childish enthusiasm."

Harry stared grimly into Professor Quirrell's eyes, who stared back without 
blinking.

"Such dangers," said Professor Quirrell coldly, "are to be discussed in offices 
like this one, not in speeches. The fools who elected Cornelius Fudge are not 
interested in complications and caution. Present them with anything more 
nuanced than a rousing cheer, and you will face your war alone. \emph{That}, 
Mr.~Potter, was your childish error, which Draco Malfoy would not have made 
even when he was eight years old. It should have been obvious even to 
\emph{you} that you should have stayed silent, and \emph{consulted with me 
first}, not spoken your worries before the crowd!"

"I am no friend of Albus Dumbledore," said Harry, a cold in his voice to match 
Professor Quirrell's. "But he is no child, and he did not seem to think my 
concerns were childish, nor that I should have waited to speak them."

"Oh," said Professor Quirrell, "so you take your cues from the Headmaster now, 
do you?" and stood up from behind his desk.
\sbreak
When Blaise turned the corner on the way to the office, he saw that Professor 
Quirrell was already leaning against the wall.

"Blaise Zabini," said the Defense Professor, straightening; his eyes were set 
like dark stones within his face, and his voice sent a shiver of fear down 
Blaise's spine.

\emph{He can't do anything against me, I just have to remember that---}

"I believe," said Professor Quirrell, in a clear, cold voice, "that I have 
already guessed the name of your employer. But I would hear it from your own 
lips, and tell me also the price that bought you."

Blaise knew he was sweating under his robes, and that the moisture would be 
already visible on his forehead. "I got a chance to show I was better than all 
three generals, and I took it. A lot of people hate me now, but there're also 
plenty of Slytherins who'll love me for it. What makes you think I'm---"

"You did not devise the plan of today's battle, Mr.~Zabini. Tell me who did."

Blaise swallowed hard. "Well{\ldots} I mean, in that case{\ldots} then you 
already know who did, right? The only one who's that crazy is Dumbledore. And 
he'll protect me if you try to do anything."

"Indeed. Tell me the price." The Defense Professor's eyes were still hard.

"It's my cousin Kimberly," Blaise said, swallowing again and trying to control 
his voice. "She's real, and she's really being bullied, Potter checked that, he 
wasn't dumb. Only Dumbledore said that he'd nudged the bullies into doing it, 
just for the plan, and if I worked for \emph{him} she'd be fine afterward, but 
if I \emph{did} go with Potter, there was more trouble Kimberly could get into!"

Professor Quirrell was silent for a long moment.

"I see," Professor Quirrell said, his voice now much milder. "Mr.~Zabini, 
should such an event occur again, you may contact me directly. I have my own 
ways of protecting my friends. Now, a final question: Even with all the power 
you took into your hands, forcing a tie would have been difficult. Did 
Dumbledore instruct you as to who should win otherwise?"

"Sunshine," said Blaise.

Professor Quirrell nodded. "As I thought." The Defense Professor sighed. "In 
your future career, Mr.~Zabini, I do not suggest trying any plots that 
complicated. They have a tendency to fail."

"Um, I said that to the Headmaster, actually," Blaise said, "and he said that 
was why it was important to have more than one plot going at a time."

Professor Quirrell passed a weary hand across his forehead. "It's a wonder the 
Dark Lord didn't go mad from fighting \emph{him.} You may go on to your meeting 
with the Headmaster, Mr.~Zabini. I will say nothing of this, but if the 
Headmaster should somehow discover that we have spoken, remember my standing 
offer to give you what protection I can. You are dismissed."

Blaise didn't wait for any other word, just turned and fled.
\sbreak
Professor Quirrell waited for a time, and then said, "Go ahead, Mr.~Potter."

Harry tore the Cloak of Invisibility off his head and stuffed into his pouch. 
He was trembing with so much rage he could hardly speak. "He \emph{what?} He 
did \emph{what?}"

"You should have deduced it yourself, Mr.~Potter," Professor Quirrell said 
mildly. "You must learn to blur your vision until you can see the forest 
obscured by the trees. Anyone who heard the stories about you, and who did not 
know that you were the mysterious Boy-Who-Lived, could easily deduce your 
ownership of an invisibility cloak. Step back from these events, blur away 
their details, and what do we observe? There was a great rivalry between 
students, and their competition ended in a perfect tie. That sort of thing only 
happens in stories, Mr.~Potter, and there is one person in this school who 
thinks in stories. There was a strange and complicated plot, which you should 
have realized was uncharacteristic of the young Slytherin you faced. But there 
is a person in this school who deals in plots that elaborate, and his name is 
not Zabini. And I did warn you that there was a quadruple agent; you knew that 
Zabini was at least a triple agent, and you should have guessed a high chance 
that it was he. No, I will not declare the battle invalid. All three of you 
failed the test, and lost to your common enemy."

Harry didn't care about tests at this point. "Dumbledore \emph{blackmailed} 
Zabini by \emph{threatening his cousin?} Just to make our battle end in a tie? 
\emph{Why?}"

Professor Quirrell gave a mirthless laugh. "Perhaps the Headmaster thought the 
rivalry was good for his pet hero and wished to see it continue. For the 
greater good, you understand. Or perhaps he was simply mad. You see, 
Mr.~Potter, everyone knows that Dumbledore's madness is a mask, that he is sane 
pretending to be insane. They pride themselves on that clever insight, and 
knowing the secret explanation, they stop looking. It does not occur to them 
that it is \emph{also} possible to have a mask behind the mask, to be insane 
pretending to be sane pretending to be insane. And I am afraid, Mr.~Potter, 
that I have urgent business elsewhere, and must depart; but I should strongly 
advise you not to take your cues from Albus Dumbledore when fighting a war. 
Until later, Mr.~Potter."

And the Defense Professor inclined his head with some irony, and then strode 
off in the same direction Zabini had fled, while Harry was still standing in 
open-mouthed shock.
\sbreak
\emph{Aftermath: Harry Potter.}

Harry trudged slowly toward the Ravenclaw dorm, eyes unseeing of walls, 
paintings, or other students; he went up stairs and down ramps without slowing, 
speeding, or noticing where he trod.

It had taken him more than a minute after Professor Quirrell's departure to 
realize that his only source of information about Dumbledore being involved was 
(a) Blaise Zabini, who he would have to be an absolute gaping idiot to trust 
again, and (b) Professor Quirrell, who could have easily faked a plot in 
Dumbledore's style, and who might also think that a little student rivalry was 
a fine thing; and who had, if you stepped back and blurred out the details, 
just proposed turning the country into a magical dictatorship.

And it was also possible that Dumbledore \emph{was} the one behind Zabini, and 
that Professor Quirrell had been sincerely trying to fight the Dark Mark in 
kind, and prevent the repetition of a performance he saw as pathetic. Trying to 
make sure that Harry didn't end up fighting the Dark Lord alone, while everyone 
else hid, frightened, trying to stay out of the line of fire, waiting for Harry 
to save them.

But the truth was{\ldots}

Well{\ldots}

Harry was sort of okay with that.

It was, he knew, the kind of thing that was supposed to make heroes resentful 
and bitter.

To heck with that. Harry was very much in favor of everyone else \emph{staying 
out of danger} while the Boy-Who-Lived took down the Dark Lord by himself, plus 
or minus a small number of companions. If the next conflict with the Dark Lord 
got to the point of a Second Wizarding War that killed lots of people and 
embroiled a whole country, that would mean Harry had \emph{already failed}.

And if afterward a war broke out between wizards and Muggles, it didn't matter 
who won, Harry would have already failed by letting it get that far. Besides, 
who said the societies couldn't peacefully integrate when the secrecy 
inevitably broke down? (Though Harry could hear Professor Quirrell's dry voice 
in his mind, asking him if he was a fool, and saying all the obvious 
things{\ldots}) And if mages and Muggles couldn't live in peace, then Harry 
would combine magic and science and figure out how to evacuate all the wizards 
to Mars or somewhere, instead of letting a war break out.

Because if it did come down to a war of extermination{\ldots}

That was the thing Professor Quirrell hadn't realized, the one most important 
question he'd forgotten to ask his young general.

The real reason why Harry had no intention of being argued into endorsing a 
Light Mark, no matter \emph{how} much it would help him in his fight against 
the Dark Lord.

One Dark Lord and fifty Marked followers had been a peril to all of magical 
Britain.

If all Britain took the Mark of a strong leader, they would be a peril to the 
whole magical world.

And if the whole wizarding world took a single Mark, they would be a danger to 
the rest of humanity.

No one knew quite how many wizards there were in the world. He'd done a few 
estimates with Hermione and come up with numbers in the rough range of a 
million.

But there were six billion Muggles.

If it came down to a final war{\ldots}

Professor Quirrell had forgotten to ask Harry which side he would protect.

A scientific civilization, reaching outward, looking upward, knowing that its 
destiny was to grasp the stars.

And a magical civilization, slowly fading as knowledge was lost, still governed 
by a nobility that saw Muggles as not quite human.

It was a terribly sad feeling, but not one that held any hint of doubt.
\sbreak
\emph{Aftermath: Blaise Zabini.}

Blaise strolled through the hallways with careful, self-imposed slowness, his 
heart beating wildly as he tried to calm down---

"Ahem," said a dry, whispering voice from a shadowy alcove as he passed.

Blaise jumped, but he didn't scream.

Slowly, he turned.

In that small, shadowy corner was a black cloak so wide and billowing that it 
was impossible to determine whether the figure beneath was male or female, and 
atop the cloak a broad-brimmed black hat, and a black mist seemed to gather 
beneath it and obscure the face of whoever or whatever might lie beneath.

"Report," whispered Mr.~Hat and Cloak.

"I said just what you told me to," said Blaise. His voice was a little calmer 
now that he wasn't lying to anyone. "And Professor Quirrell reacted just the 
way you expected."

The broad black hat tilted and straightened, as though the head below had 
nodded. "Excellent," said the unidentifiable whisper. "The reward I promised 
you is already on its way to your mother, by owl."

Blaise hesitated, but his curiosity was eating him alive. "Can I ask now why 
you want to cause trouble between Professor Quirrell and Dumbledore?" The 
Headmaster hadn't had anything to do with the Gryffindor bullies that Blaise 
knew about, and besides helping Kimberly, the Headmaster had also offered to 
make Professor Binns give him excellent marks in History of Magic even if he 
turned in blank parchments for his homework, though he'd still have to attend 
class and pretend to hand them in. Actually Blaise would have betrayed all 
three generals for free, and never mind his cousin either, but he'd seen no 
need to say that.

The broad black hat cocked to one side, as if to convey a quizzical stare. 
"Tell me, friend Blaise, did it occur to you that traitors who betray so many 
times over often meet with ill ends?"

"Nope," said Blaise, looking straight into the black mist under the hat. 
"Everyone knows that nothing \emph{really} bad ever happens to students in 
Hogwarts."

Mr.~Hat and Cloak gave a whispery chuckle. "Indeed," said the whisper. "With 
the murder of one student five decades ago being the exception that proves the 
rule, since Salazar Slytherin would have keyed his monster into the ancient 
wards at a higher level than the Headmaster himself."

Blaise stared at the black mist, now beginning to feel a little uneasy. But it 
ought to take a Hogwarts professor to do anything significant to him without 
setting off alarms. Quirrell and Snape were the only professors who'd do 
something like this, and Quirrell wouldn't care about fooling \emph{himself}, 
and Snape wouldn't hurt one of his own Slytherins{\ldots} would he?

"No, friend Blaise," whispered the black mist, "I only wished to advise you 
never to try anything like this in your adult life. So many betrayals would 
certainly lead to at least one vengeance."

"My \emph{mother} never got any vengeances," said Blaise proudly. "Even though 
she married \emph{seven} husbands and every single one of them died 
mysteriously and left her lots of money."

"Really?" said the whisper. "However did she persuade the seventh to marry her 
after he heard what happened to the first six?"

"I asked Mum that," said Blaise, "and she said I couldn't know until I was old 
enough, and I asked her how old was old enough, and she said, older than her."

Again the whispery chuckle. "Well then, friend Blaise, my congratulations on 
having followed in your mother's footsteps. Go, and if you say nothing of this, 
we will not meet again."

Blaise backed uneasily away, feeling an odd reluctance to turn his back.

The hat tilted. "Oh, come now, little Slytherin. If you were truly the equal of 
Harry Potter or Draco Malfoy, you would have already realized that my hinted 
threats were just to ensure your silence before Albus. Had I intended to harm, 
I would not have hinted; had I said nothing, \emph{then} you should have 
worried."

Blaise straightened, feeling a little insulted, and nodded to Mr.~Hat and 
Cloak; then turned decisively and strode off toward his meeting with the 
Headmaster.

He'd been hoping to the very end that someone \emph{else} would show up and 
give him a chance to sell out Mr.~Hat and Cloak.

But then Mum hadn't betrayed seven different husbands at the \emph{same time.} 
When you looked at it \emph{that} way, he was still doing better than her.

And Blaise Zabini went on walking toward the Headmaster's office, smiling, 
content to be a quintuple agent---

For a moment the boy stumbled, but then straightened, shaking off the odd 
feeling of disorientation.

And Blaise Zabini went on walking toward the Headmaster's office, smiling, 
content to be a quadruple agent.
\sbreak
\emph{Aftermath: Hermione Granger.}

The messenger didn't approach her until she was alone.

Hermione was just leaving the girl's bathroom where she sometimes hid to think, 
and a bright shining cat leaped out of nowhere and said, "Miss Granger?"

She let out a little shriek before she realized the cat had spoken in Professor 
McGonagall's voice.

Even so she hadn't been frightened, only startled; the cat was bright and 
brilliant and beautiful, glowing with a white silver radiance like moon-colored 
sunlight, and she couldn't imagine being scared.

"What are you?" said Hermione.

"This is a message from Professor McGonagall," said the cat, still in the 
Professor's voice. "Can you come to my office, and not speak of this to anyone?"

"I'll be there right away," said Hermione, still surprised, and the cat leaped 
and vanished; only it didn't vanish, it traveled away somehow; or that was what 
her mind said, even though her eyes just saw it disappear.

By the time Hermione had got to the office of her favorite professor, her mind 
was all a-whirl with speculations. Was there something wrong with her 
Transfiguration scores? But then why would Professor McGonagall say not to tell 
anyone? It was probably about Harry practicing his partial 
Transfiguration{\ldots}

Professor McGonagall's face looked worried, not stern, as Hermione seated 
herself in front of the desk---trying to keep her eyes from going to the nest 
of cubbyholes containing Professor McGonagall's homework, she'd always wondered 
what sort of work grownups had to do to keep the school running and whether 
they could use any help from her{\ldots}

"Miss Granger," said Professor McGonagall, "let me start by saying that I 
already know about the Headmaster asking you to make that wish---"

"He \emph{told} you?" blurted Hermione in startlement. The Headmaster had said 
no one else was supposed to know!

Professor McGonagall paused, looked at Hermione, and gave a sad little chuckle. 
"It's good to see Mr.~Potter hasn't corrupted you too much. Miss Granger, you 
aren't supposed to \emph{admit} anything just because I say I know. As it 
happens, the Headmaster did \emph{not} tell me, I simply know him too well."

Hermione was blushing furiously now.

"It's fine, Miss Granger!" said Professor McGonagall hastily. "You're a 
Ravenclaw in your first year, nobody expects you to be a Slytherin."

That \emph{really} stung.

"Fine," said Hermione with some acerbity, "I'll go ask Harry Potter for 
Slytherin lessons, then."

"That \emph{wasn't} what I wanted to{\ldots}" said Professor McGonagall, and 
her voice trailed off. "Miss Granger, I'm worried about this \emph{because} 
young Ravenclaw girls shouldn't have to be Slytherins! If the Headmaster asks 
you to get involved in something you're not comfortable with, Miss Granger, it 
really is all right to say no. And if you're feeling pressured, please tell the 
Headmaster that you would like me to be there, or that you would like to ask me 
first."

Hermione's eyes were very wide. "Does the Headmaster do things that are wrong?"

Professor McGonagall looked a little sad at that. "Not on purpose, Miss 
Granger, but I think{\ldots} well, it probably \emph{is} true that sometimes 
the Headmaster has trouble remembering what it's like to be a child. Even when 
he was a child, I'm sure he must have been brilliant, and strong of mind and 
heart, with courage enough for three Gryffindors. Sometimes the Headmaster asks 
too much of his young students, Miss Granger, or isn't careful enough not to 
hurt them. He is a good man, but sometimes his plotting can go too far."

"But it's \emph{good} for students to be strong and have courage," said 
Hermione. "That's why you suggested Gryffindor for me, wasn't it?"

Professor McGonagall smiled wryly. "Perhaps I was only being selfish, wanting 
you for my own House. Did the Sorting Hat offer you---no, I should not have 
asked."

"It told me I might go anywhere but Slytherin," said Hermione. She'd 
\emph{almost} asked why she wasn't good enough for Slytherin, before she'd 
managed to stop herself{\ldots} "So I \emph{have} courage, Professor!"

Professor McGonagall leaned forward over her desk. The worry was showing 
plainer on her face now. "Miss Granger, it's not about courage, it's about 
what's healthy for young girls! The Headmaster is drawing you into his plots, 
Harry Potter is giving you his secrets to keep, and now you're making alliances 
with Draco Malfoy! And I promised your mother that you would be safe at 
Hogwarts!"

Hermione just didn't know what to say to that. But the thought was occurring to 
her that Professor McGonagall might not have been warning her if she'd been a 
boy in Gryffindor instead of a girl in Ravenclaw and \emph{that} was, 
well{\ldots} "I'll try to be good," she said, "and I won't let anyone tell me 
otherwise."

Professor McGonagall pressed her hands over her eyes. When she took them away, 
her lined face looked very old. "Yes," she said in a whisper, "you would have 
done well in my House. Stay safe, Miss Granger, and be careful. And if you are 
ever worried or uncomfortable about anything, please come to me at once. I 
won't keep you any longer."
\sbreak
\emph{Aftermath, Draco Malfoy:}

Neither of them really wanted to do anything complicated that Saturday, not 
after fighting a battle earlier. So Draco was just sitting in an unused 
classroom and trying to read a book called \emph{Thinking Physics.} It was one 
of the most fascinating things that Draco had ever read in his life, at least 
the parts he could understand, at least when the \emph{accursed idiot} who 
refused to let his books out of his sight could manage to \emph{shut up} and 
let Draco \emph{concentrate}---

"Hermione Granger is a \emph{muuudbloood,}" sang Harry Potter from where he sat 
at a nearby desk, reading a far more advanced book of his own.

"I know what you're trying to do," said Draco calmly without looking up from 
the pages. "It's not going to work. We're still ganging up and crushing you."

"A \emph{Maaaalfoy} is working with a \emph{muuudbloood,} what will all your 
father's \emph{frieeeends} think---"

"They'll think Malfoys aren't as easily manipulated as \emph{you} seem to 
believe, \emph{Potter!}"

The Defense Professor was crazier than Dumbledore, no future saviour of the 
world could ever be this \emph{childish} and \emph{undignified} at any age.

"Hey, Draco, you know what's really going to suck? \emph{You} know that 
Hermione Granger has two copies of the magical allele, just like you and just 
like me, but all your classmates in Slytherin don't know that and 
\emph{yooouuu're} not allowed to \emph{explaaaaain}---"

Draco's fingers were whitening where they gripped the book. Being beaten and 
spat upon couldn't possibly require this much self-control, and if he didn't 
get back at Harry soon, he was going to do something incriminating---

"So what \emph{did} you wish the first time?" said Draco.

Harry didn't say anything, so Draco looked up from his book, and felt a twinge 
of malicious satisfaction at the sad look on Harry's face.

"Um," Harry said. "A lot of people asked me that, but I don't think Professor 
Quirrell would have wanted me to talk about it."

Draco put a serious look on his own face. "You can talk about it with 
\emph{me}. It's probably not important compared to the other secrets you've 
told me, and what else are friends for?" \emph{That's right, I'm your friend! 
Feel guilty!}

"It wasn't really all that interesting," Harry said with obviously artificial 
lightness. "Just, \emph{I wish Professor Quirrell would teach Battle Magic 
again next year.}"

Harry sighed, and looked back down at his book.

And said, after another few seconds, "Your father's probably going to be pretty 
upset with you this Christmas, but if you promise him that you'll betray the 
mudblood girl and wipe out her army, everything will go back to being all 
right, and you'll still get your Christmas presents."

Maybe if he and Granger asked Professor Quirrell extra politely and used some 
of their Quirrell points, the two of them would be allowed to do something more 
interesting to General Chaos than putting him to sleep.