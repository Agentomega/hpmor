
Chapter 25: Hold Off on Proposing Solutions\\
</div><div  class='storycontent nocopy' id='storycontent' ><p>To seek out new 
life, and J. K. Rowling!

Note: Since the science in this story is usually all correct, I include a 
warning that in Ch. 22-25 Harry overlooks many possibilities, the most 
important of which is that there are lots of magical genes but they're all on 
one chromosome (which wouldn't happen naturally, but the chromosome might have 
been engineered). In this case, the inheritance pattern would be Mendelian, but 
the magical chromosome could still be degraded by chromosomal crossover with 
its nonmagical homologue. (Harry has read about Mendel and chromosomes in 
science history books, but he hasn't studied enough actual genetics to know 
about chromosomal crossover. Hey, he's only eleven.) However, although a modern 
science journal would find a \emph{lot} more nits to pick, everything Harry 
presents as strong evidence is in fact strong evidence---the other 
possibilities are \emph{improbable}.
\sbreak
\emph{Act 2:}

(The sun shone brilliantly into the Great Hall from the enchanted sky-ceiling 
above, illuminating the students as though they sat beneath the naked sky, 
gleaming from their plates and bowls, as, refreshed by a night's sleep, they 
inhaled breakfast in preparation for whatever plans they'd made for their 
Sunday.)

So. There was only one thing that made you a wizard.

That wasn't surprising, when you thought about it. What DNA mostly did was tell 
ribosomes how to chain amino acids together into proteins. Conventional physics 
seemed quite capable of describing amino acids, and no matter how many amino 
acids you chained together, conventional physics said you would never, ever get 
magic out of it.

And yet magic seemed to be hereditary, following DNA.

Then that probably\emph{ wasn't} because the DNA was chaining together 
nonmagical amino acids into magical proteins.

Rather the key DNA sequence did not, of itself, give you your magic at all.

Magic came from somewhere else.

(At the Ravenclaw table there was one boy who was staring off into space, as 
his right hand automatically spooned some unimportant food into his mouth from 
whatever was in front of him. You probably could have substituted a pile of 
dirt and he wouldn't have noticed.)

And for some reason the Source of Magic was paying attention to a particular 
DNA marker among individuals who were ordinary ape-descended humans in every 
other way.

(Actually there were quite a lot of boys and girls staring off into space. It 
was the \emph{Ravenclaw} table, after all.)

There were other lines of logic leading to the same conclusion. \emph{Complex} 
machinery was always universal within a sexually reproducing species. If gene B 
relied on gene A, then A had to be useful on its own, and rise to 
near-universality in the gene pool on its own, before B would be useful often 
enough to confer a fitness advantage. Then once B was universal you would get a 
variant A* that relied on B, and then C that relied on A* and B, then B* that 
relied on C, until the whole machine would fall apart if you removed a single 
piece. But it all had to happen \emph{incrementally}---evolution never looked 
ahead, evolution would never start promoting B in \emph{preparation} for A 
becoming universal later. Evolution was the simple historical fact that, 
whichever organisms did in fact have the most children, their genes would in 
fact be more frequent in the next generation. So each piece of a complex 
machine had to become nearly universal before other pieces in the machine would 
evolve to depend on its presence.

So \emph{complex, interdependent} machinery, the powerful sophisticated protein 
machines that drove life, was always \emph{universal} within a sexually 
reproducing species---except for a small handful of \emph{non}-interdependent 
\emph{variants} that were being selected on at any given time, as further 
complexity was slowly laid down. It was why all human beings had the same 
underlying brain design, the same emotions, the same facial expressions wired 
up to those emotions; those adaptations were complex, so they \emph{had} to be 
universal.

If magic had been like that, a big complex adaptation with lots of necessary 
genes, then a wizard mating with a Muggle would have resulted in a child with 
only half those parts and half the machine wouldn't do much. And so there would 
have been no Muggleborns, ever. Even if all the pieces had individually gotten 
into the Muggle gene pool, they'd never reassemble all in one place to form a 
wizard.

There hadn't been some genetically isolated valley of humans that had stumbled 
onto an evolutionary pathway leading to sophisticated magical sections of the 
brain. That complex genetic machinery, if wizards interbred with Muggles, would 
never have reassembled into Muggleborns.

So however your genes made you a wizard, it \emph{wasn't} by containing the 
blueprints for complicated machinery.

That was the other reason Harry had guessed the Mendelian pattern would be 
there. If magical genes weren't complicated, why would there be more than one?

And yet magic itself seemed pretty complicated. A door-locking spell would 
prevent the door from opening \emph{and} prevent you from Transfiguring the 
hinges \emph{and} resist \emph{Finite Incantatem} and \emph{Alohomora}. Many 
elements all pointing in the same direction: you could call that 
goal-orientation, or in simpler language, purposefulness.

There were only two known causes of purposeful complexity. Natural selection, 
which produced things like butterflies. And intelligent engineering, which 
produced things like cars.

Magic didn't seem like something that had self-replicated into existence. 
Spells were purposefully complicated, but not, like a butterfly, complicated 
for the purpose of making copies of themselves. Spells were complicated for the 
purpose of serving their user, like a car.

Some intelligent engineer, then, had created the Source of Magic, and told it 
to pay attention to a particular DNA marker.

The obvious next thought was that this had something to do with "Atlantis".

Harry had asked Hermione about that earlier---on the train to Hogwarts, after 
hearing Draco say it---and so far as she knew, nothing more was known than the 
word itself.

It might have been pure legend. But it was also plausible enough that a 
civilization of magic-users, especially one from \emph{before} the Interdict of 
Merlin, would have managed to blow itself up.

The line of reasoning continued: Atlantis had been an isolated civilization 
that had somehow brought into being the Source of Magic, and told it to serve 
only people with the Atlantean genetic marker, the blood of Atlantis.

And by similar logic: The words a wizard spoke, the wand movements, those 
weren't complicated enough of themselves to build up the spell effects from 
scratch---not the way that the three billion base pairs of human DNA actually 
\emph{were} complicated enough to build a human body from scratch, not the way 
that computer programs took up thousands of bytes of data.

So the words and wand movements were just triggers, levers pulled on some 
hidden and more complex machine. Buttons, not blueprints.

And just like a computer program wouldn't compile if you made a single spelling 
error, the Source of Magic wouldn't respond to you unless you cast your spells 
in exactly the right way.

The chain of logic was inexorable.

And it led inevitably toward a single final conclusion.

The ancient forebears of the wizards, thousands of years earlier, had told the 
Source of Magic to only levitate things if you said{\ldots}

'Wingardium Leviosa.'

Harry slumped over at the breakfast table, resting his forehead wearily on his 
right hand.

There was a story from the dawn days of Artificial Intelligence---back when 
they were just starting out and no one had yet realized the problem would be 
difficult---about a professor who had delegated one of his grad students to 
solve the problem of computer vision.

Harry was beginning to understand how that grad student must have felt.

This could take a while.

Why did it take more effort to cast the Alohomora spell, if it was just like 
pressing a button?

Who'd been silly enough to build in a spell for \emph{Avada Kedavra} that could 
only be cast using hatred?

Why did wordless Transfiguration require you to make a complete mental 
separation between the concept of form and concept of material?

Harry might not be done with this problem by the time he graduated Hogwarts. He 
could still be working on this problem when he was \emph{thirty years old.} 
Hermione had been right, Harry \emph{hadn't} realized that on a gut level 
before. He'd just given an inspiring speech about determination.

Harry's mind briefly considered whether to get on a gut level that he might 
never solve the problem at all, then decided that would be taking things much 
too far.

Besides, so long as he could get as far as immortality in the first few 
decades, he'd be fine.

What method had the Dark Lord used? Come to think, the fact that the Dark Lord 
had somehow managed to survive the death of his first body was almost 
\emph{infinitely} more important than the fact that he'd tried to take over 
magical Britain---

"Excuse me," said an expected voice from behind him in very unexpected tones. 
"At your convenience, Mr.~Malfoy requests the favor of a conversation."

Harry did not choke on his breakfast cereal. Instead he turned around and 
beheld Mr.~Crabbe.

"Excuse \emph{me,}" said Harry. "Don't you mean 'Da boss wants ta talk wid 
youse?'"

Mr.~Crabbe didn't look happy. "Mr.~Malfoy instructed me to speak properly."

"I can't hear you," Harry said. "You're not speaking properly." He turned back 
to his bowl of tiny blue crystal snowflakes and deliberately ate another 
spoonful.

"Da boss wants to talk with youse," came a threatening voice from behind him. 
"Ya'd better come see him if ya know what's good for ya."

There. \emph{Now} everything was going according to plan.
\sbreak
\emph{Act 1:}

"A \emph{reason?}" said the old wizard. He restrained the fury from his face. 
The boy before him had been the victim, and certainly did not need to be 
frightened any further. "There is \emph{nothing} that can excuse---"

"What I did to him was worse."

The old wizard stiffened in sudden horror. "Harry, \emph{what have you done?}"

"I tricked Draco into believing that I'd tricked him into participating in a 
ritual that sacrificed his belief in blood purism. And that meant he couldn't 
be a Death Eater when he grew up. He'd lost everything, Headmaster."

There was a long quiet in the office, broken only by the tiny puffs and 
whistles of the fiddly things, which after enough time had come to seem like 
silence.

"Dear me," said the old wizard, "I \emph{do} feel silly. And \emph{here} I was 
expecting you might try to redeem the heir of Malfoy by, say, \emph{showing him 
true friendship and kindness}."

"\emph{Ha!} Yeah, like \emph{that} would have worked."

The old wizard sighed. This was taking it too far. "Tell me, Harry. Did it even 
\emph{occur} to you that there was something \emph{incongruous} about setting 
out to redeem someone through lies and trickery?"

"I did it without telling any direct lies, and since we're talking about Draco 
Malfoy here, I think the word you're looking for is \emph{congruous}." The boy 
looked rather smug.

The old wizard shook his head in despair. "And \emph{this} is the hero. We're 
all doomed."
\sbreak
\emph{Act 5:}

The long, narrow tunnel of rough stone, unlit except by a child's wand, seemed 
to stretch on for miles.

The reason for this was simple: It \emph{did} stretch on for miles.

The time was three in the morning, and Fred and George were starting the long 
way down the secret passage that led from a statue of a one-eyed witch in 
Hogwarts, to the cellar of the Honeydukes candyshop in Hogsmeade.

"How's it doing?" said Fred in a low voice.

(Not that there'd be anyone listening, but there was something odd about 
talking in a normal voice when you were going through a secret passage.)

"Still on the fritz," said George.

"Both, or---"

"Intermittent one fixed itself again. Other one's same as ever."

The Map was an extraordinarily powerful artifact, capable of tracking every 
sentient being on the school grounds, in real time, by name. Almost certainly, 
it had been created during the original raising of Hogwarts. It was \emph{not 
good} that errors were starting to pop up. Chances were that no one except 
Dumbledore could fix it if it was broken.

And the Weasley twins weren't about to turn the Map over to Dumbledore. It 
would have been an unforgivable insult to the Marauders---the four unknowns 
who'd managed to steal part of the \emph{Hogwarts security system}, something 
probably forged by Salazar Slytherin himself, and twist it into \emph{a tool 
for student pranking}.

Some might have considered it disrespectful.

Some might have considered it criminal.

The Weasley twins firmly believed that if Godric Gryffindor had been around to 
see it, he would have approved.

The brothers walked on and on and on, mostly in silence. The Weasley twins 
talked to each other when they were thinking through new pranks, or when one of 
them knew something the other didn't. Otherwise there wasn't much point. If 
they already knew the same information, they tended to think the same thoughts 
and make the same decisions.

(Back in the old days, whenever magical identical twins were born, it had been 
the custom to kill one of them after birth.)

In time, Fred and George clambered out into a dusty cellar, strewn with barrels 
and racks of strange ingredients.

Fred and George waited. It wouldn't have been polite to do anything else.

Before too long a thin old man in black pajamas clambered down the steps that 
led into the cellar, yawning. "Hello, boys," said Ambrosius Flume. "I wasn't 
expecting you tonight. Out of stock already?"

Fred and George decided that Fred would speak.

"Not exactly, Mr.~Flume," said Fred. "We were hoping you could help us with 
something considerably more{\ldots} interesting."

"Now, boys," said Flume, sounding severe, "I hope you didn't wake me up just so 
I could tell you again that I'm not selling you any merchandise that could get 
you into real trouble. Not until you're sixteen, anyways---"

George drew forth an item from his robes, and wordlessly passed it to Flume. 
"Have you seen this?" said Fred.

Flume looked at yesterday's edition of the \emph{Daily Prophet} and nodded, 
scowling. The headline on the paper read THE NEXT DARK LORD? and showed a young 
boy which some student's camera had managed to catch in an uncharacteristically 
cold and grim expression.

"I can't believe that Malfoy," Flume snapped. "Going after the boy when he's 
only eleven! The man ought to be ground up and used to make chocolates!"

Fred and George blinked in unison. \emph{Malfoy} was behind Rita Skeeter? Harry 
Potter hadn't warned them about that{\ldots} which surely meant that Harry 
didn't know. He never would have brought them in if he did{\ldots}

Fred and George exchanged glances. Well, Harry didn't \emph{need} to know until 
after the job was done.

"Mr.~Flume," Fred said quietly, "the Boy-Who-Lived needs your help."

Flume looked at them both.

Then he let out his breath with a sigh.

"All right," said Flume, "what do you want?"
\sbreak
\emph{Act 6:}

When Rita Skeeter was intent on a tasty prey, she didn't tend to notice the 
scurrying ants who constituted the rest of the universe, which was how she 
almost bumped into the balding young man who'd stepped into her pathway.

"Miss Skeeter," said the man, sounding rather severe and cold for someone whose 
face looked that young. "Fancy running into you here."

"Out of my way, buster!" snapped Rita, and tried to step around him.

The man in her pathway matched the movement so perfectly that it was like 
neither of them had moved at all, just stood still while the street shifted 
around them.

Rita's eyes narrowed. "Who do you think you are?"

"How very foolish," the man said dryly. "It would have been wise to memorize 
the face of the disguised Death Eater training Harry Potter to be the next Dark 
Lord. After all," a thin smile, "\emph{that} certainly sounds like someone you 
wouldn't want to run into on the street, especially after doing a hatchet job 
on him in the newspaper."

Rita took a moment to place the reference. \emph{This} was Quirinus Quirrell? 
He looked too young and too old at the same time; his face, if it relaxed from 
its severe and condescending pose, would belong to someone in his late 
thirties. And his hair was already falling out? Couldn't he afford a healer?

No, that wasn't important, she had a time and a place and a beetle to be. She'd 
just received an anonymous tip about Madam Bones making time with one of her 
younger assistants. That would be worth quite a bonus if she could manage to 
verify it, Bones was high on the hit list. The tipster had said that Bones and 
her young assistant were due to eat lunch in a special room at Mary's Place, a 
very popular room for certain purposes; a room which, she'd found, was secure 
against all listening devices, but not proof against a beautiful blue beetle 
nestled up against one wall{\ldots}

"Out of my \emph{way!}" Rita said, and tried to push Quirrell from her path. 
Quirrell's arm brushed her own, deflecting, and Rita staggered as the thrust 
went into the thin air.

Quirrell pulled up the sleeve of his left robe, showing his left arm. 
"Observe," said Quirrell, "no Dark Mark. I would like your paper to publish a 
retraction."

Rita let out an incredulous laugh. Of course the man wasn't a real Death Eater. 
The paper wouldn't have published it if he was. "Forget it, buster. Now take a 
hike."

Quirrell stared at her for a moment.

Then he smiled.

"Miss Skeeter," said Quirrell, "I had hoped to find some lever that would prove 
persuasive. Yet I find that I cannot deny myself the pleasure of simply 
crushing you."

"It's been tried. Now get out of my way, buster, or I'll find some Aurors and 
have you arrested for obstruction of journalism."

Quirrell swept her a small bow, and then walked past. "Goodbye, Rita Skeeter," 
said his voice from behind her.

As Rita bulled on ahead, she noted in the back of her mind that the man was 
whistling a tune as he walked away.

Like \emph{that} would scare her.
\sbreak
\emph{Act 4:}

"Sorry, count me out," said Lee Jordan. "I'm more the giant spider type."

The Boy-Who-Lived had said that he had \emph{important} work for the Order of 
Chaos, something serious and secret, more significant and difficult than their 
usual run of pranks.

And then Harry Potter had launched into a speech that was inspiring, yet vague. 
A speech to the effect that Fred and George and Lee had tremendous potential if 
they could just learn to be \emph{weirder.} To make people's lives 
\emph{surreal,} instead of just surprising them with the equivalents of buckets 
of water propped above doors. (Fred and George had exchanged interested looks, 
they'd never thought of that one.) Harry Potter had invoked a picture of the 
prank they'd pulled on Neville---which, Harry had mentioned with some remorse, 
the Sorting Hat had chewed him out on---but which must have made Neville 
\emph{doubt his own sanity.} For Neville it would have felt like being suddenly 
transported into an alternate universe. The same way everyone else had felt 
when they'd seen Snape apologize. That was the \emph{true power of pranking.}

\emph{Are you with me?} Harry Potter had cried, and Lee Jordan had answered no.

"Count us \emph{in}," said Fred, or possibly George, for there was no doubt 
that Godric Gryffindor would have said yes.

Lee Jordan gave a regretful grin, and stood up, and left the deserted and 
Quieted corridor where the four members of the Order of Chaos had met and sat 
down in a conspiratorial circle.

The three members of the Order of Chaos got down to business.

(It wasn't \emph{that} sad. Fred and George would still work with Lee on giant 
spider pranks, same as ever. They'd only started calling it the Order of Chaos 
in order to recruit Harry Potter, after Ron had told them about Harry being 
weird and evil, and Fred and George had decided to save Harry by showing him 
true friendship and kindness. Thankfully this no longer seemed 
necessary---although they weren't \emph{quite} sure about that{\ldots})

"So," said one of the twins, "what's this about?"

"Rita Skeeter," said Harry. "Do you know who she is?"

Fred and George nodded, frowning.

"She's been asking questions about me."

That wasn't good news.

"Can you guess what I want you to do?"

Fred and George looked at each other, a bit puzzled. "You want us to slip her 
some of our more interesting candies?"

"No," said Harry. "No, no, \emph{no!} That's giant-spider thinking! Come on, 
what would \emph{you} do if you heard that Rita Skeeter was looking for rumors 
about \emph{you?}"

That made it obvious.

Grins slowly started on the faces of Fred and George.

"Start rumors about ourselves," they replied.

"\emph{Exactly,}" said Harry, grinning widely. "But these can't be just 
\emph{any} rumors. I want to teach people never to believe what the newspaper 
says about Harry Potter, any more than Muggles believe what the newspaper says 
about Elvis. At first I just thought about flooding Rita Skeeter with so many 
rumors that she wouldn't know what to believe, but then she'll just cherry-pick 
the ones that sound plausible and bad. So what I want you to do is create a 
fake story about me, and get Rita Skeeter to believe it somehow. But it has to 
be something that, afterward, everyone will \emph{know} was fake. We want to 
fool Rita Skeeter and her editors, and \emph{afterward} have the proof come out 
that it was false. And of course---given that those are the requirements---the 
story has to be as \emph{ridiculous} as it can possibly be, and still get 
printed. Do you understand what I want you to do?"

"Not exactly{\ldots}" Fred or George said slowly. "You want us to \emph{invent} 
the story?"

"I want you to do \emph{all} of it," Harry Potter said. "I'm sort of busy right 
now, plus I want to be able to say truthfully that I had no idea what was 
coming. Surprise me."

For a moment there was a very evil grin on the faces of Fred and George.

Then they turned serious. "But Harry, we don't really know how to do anything 
like that---"

"So figure it out," Harry said. "I have confidence in you. Not \emph{total} 
confidence, but if you \emph{can't} do it, \emph{tell} me that, and I'll try 
someone else, or do it myself. If you have a really good idea---for both the 
ridiculous story, and how to convince Rita Skeeter and her editors to print 
it---then you can go ahead and do it. But don't go with something mediocre. If 
you can't come up with something \emph{awesome}, just say so."

Fred and George exchanged worried glances.

"I can't think of anything," said George.

"Neither can I," said Fred. "Sorry."

Harry stared at them.

And then Harry began to explain how you went about thinking of things.

It had been known to take longer than two seconds, said Harry.

You \emph{never} called \emph{any} question impossible, said Harry, until you 
had taken an actual clock and thought about it for five minutes, by the motion 
of the minute hand. Not five minutes metaphorically, five minutes by a physical 
clock.

And \emph{furthermore,} Harry said, his voice emphatic and his right hand 
thumping hard on the floor, you did \emph{not} start out immediately looking 
for solutions.

Harry then launched into an explanation of a test done by someone named Norman 
Maier, who was something called an organizational psychologist, and who'd asked 
two different sets of problem-solving groups to tackle a problem.

The problem, Harry said, had involved three employees doing three jobs. The 
junior employee wanted to just do the easiest job. The senior employee wanted 
to rotate between jobs, to avoid boredom. An efficiency expert had recommended 
giving the junior person the easiest job and the senior person the hardest job, 
which would be 20% more productive.

\emph{One} set of problem-solving groups had been given the instruction "Do not 
propose solutions until the problem has been discussed as thoroughly as 
possible without suggesting any."

The other set of problem-solving groups had been given no instructions. And 
those people had done the natural thing, and reacted to the presence of a 
problem by proposing solutions. And people had gotten attached to those 
solutions, and started fighting about them, and arguing about the relative 
importance of freedom versus efficiency and so on.

The first set of problem-solving groups, the ones given instructions to 
\emph{discuss} the problem first and \emph{then} solve it, had been far more 
likely to hit upon the solution of letting the junior employee keep the easiest 
job and rotating the other two people between the other two jobs, for what the 
expert's data said would be a 19% improvement.

Starting out by looking for solutions was taking things\emph{ entirely out of 
order.} Like starting a meal with dessert, only \emph{bad.}

(Harry also quoted someone named Robyn Dawes as saying that the harder a 
problem was, the more likely people were to try to solve it immediately.)

So Harry was going to leave this problem to Fred and George, and they would 
discuss all the aspects of it and brainstorm anything they thought might be 
remotely relevant. And they shouldn't try to come up with an actual solution 
until they'd finished doing that, unless of course they \emph{did} happen to 
randomly think of something awesome, in which case they could write it down for 
afterward and then go back to thinking. And he didn't want to hear back from 
them about any so-called \emph{failures to think of anything} for at least a 
week. Some people spent \emph{decades} trying to think of things.

"Any questions?" said Harry.

Fred and George stared at each other.

"I can't think of any."

"Neither can I."

Harry coughed gently. "You didn't ask about your budget."

\emph{Budget?} they thought.

"I could just tell you the amount," Harry said. "But I think \emph{this} will 
be more \emph{inspiring}."

Harry's hands dipped into his robe, and brought forth---

Fred and George almost fell over, even though they were sitting down.

"Don't spend it for the sake of spending it," Harry said. On the stone floor in 
front of them gleamed an absolutely ridiculous amount of money. "Only spend it 
if awesomeness requires; and what awesomeness does require, don't hesitate to 
spend. If there's anything left over, just return it afterward, I trust you. 
Oh, and you get ten percent of what's there, regardless of how much you end up 
spending---"

"We \emph{can't!}" blurted one of the twins. "We don't accept money for that 
sort of thing!"

(The twins never took money for doing anything illegal. Unknown to Ambrosius 
Flume, they were selling all of his merchandise at zero percent markup. Fred 
and George wanted to be able to testify---under Veritaserum if necessary---that 
they had not been profiteering criminals, just providing a public service.)

Harry frowned at them. "But I'm asking you to put in some real work here. A 
grownup would get paid for doing something like this, and it would \emph{still} 
count as a favor for a friend. You can't just hire people for this sort of 
thing."

Fred and George shook their heads.

"Fine," Harry said. "I'll just get you expensive Christmas presents, and if you 
try returning them to me I'll burn them. Now you don't even \emph{know} how 
much I'm going to spend on you, except, obviously, that it's going to be more 
than if you'd just taken the money. And I'm going to buy you those presents 
\emph{anyway,} so think about \emph{that} before you tell me \emph{you can't 
think of anything awesome}."

Harry stood up, smiling, and turned to go while Fred and George were still 
gaping in shock. He strode a few steps away, and then turned back.

"Oh, one last thing," Harry said. "Leave Professor Quirrell out of whatever you 
do. He doesn't like publicity. I know it'd be easier to get people to believe 
weird things about the Defense Professor than anyone else, and I'm sorry to 
have to get in your way like that, but please, leave Professor Quirrell out of 
it."

And Harry turned again and took a few more steps---

Looked back one last time, and said, softly, "Thank you."

And left.

There was a long pause after he'd departed.

"So," said one.

"So," said the other.

"The Defense Professor doesn't like publicity, does he."

"Harry doesn't know us very well, does he."

"No, he doesn't."

"But we won't use his money for that, of course."

"Of course not, that wouldn't be right. We'll do the Defense Professor 
separately."

"We'll get some Gryffindors to write Skeeter, and say{\ldots}"

"{\ldots}his sleeve lifted up one time in Defense class, and they saw the Dark 
Mark{\ldots}"

"{\ldots}and he's probably teaching Harry Potter all sorts of dreadful 
things{\ldots}"

"{\ldots}and he's the worst Defense Professor anyone remembers even in 
Hogwarts, he's not just \emph{failing} to teach us, he's getting everything 
wrong, the complete opposite of what it should be{\ldots}"

"{\ldots}like when he claimed that you could only cast the Killing Curse using 
love, which made it pretty much useless."

"I like that one."

"Thanks."

"I bet the Defense Professor likes it too."

"He does have a sense of humor. He wouldn't have called us what he did if he 
didn't have a sense of humor."

"But are we really going to be able to do Harry's job?"

"Harry said to discuss the problem before trying to solve it, so let's do that."

The Weasley twins decided that George would be the enthusiastic one while Fred 
doubted.

"It all seems sort of contradictory," said Fred. "He wants it to be ridiculous 
enough that everyone laughs at Skeeter and knows it's wrong, and he wants 
Skeeter to believe it. We can't do both things at the same time."

"We'll have to fake up some evidence to convince Skeeter," said George.

"Was that a solution?" said Fred.

They considered this.

"Maybe," said George, "but I don't think we should be all \emph{that} strict 
about it, do you?"

The twins shrugged helplessly.

"So then the fake evidence has to be good enough to convince Skeeter," said 
Fred. "Can we really do that on our own?"

"We don't have to do it on our own," said George, and pointed to the pile of 
money. "We can hire other people to help us."

The twins got a thoughtful look on their face.

"That could use up Harry's budget pretty fast," said Fred. "This is a lot of 
money for us, but it's not a lot of money for someone like Flume."

"Maybe people will give discounts if they know it's for Harry," said George. 
"But most importantly of all, whatever we do, it has to be \emph{impossible}."

Fred blinked. "What do you mean, \emph{impossible?}"

"So impossible that we don't get in trouble, because no one believes we could 
have done it. So impossible that even Harry starts wondering. It has to be 
surreal, it has to make people doubt their own sanity, it has to be{\ldots} 
\emph{better than Harry.}"

Fred's eyes were wide in astonishment. This happened sometimes, between them, 
but not often. "But why?"

"They were pranks. They were \emph{all} pranks. The pie was a prank. The 
Remembrall was a prank. Kevin Entwhistle's cat was a prank. \emph{Snape} was a 
prank. \emph{We're} the best pranksters in Hogwarts, are we going to roll over 
and give up without a fight?"

"He's the Boy-Who-Lived," said Fred.

"And \emph{we're} the Weasley twins! He's \emph{challenging} us. He said we 
could do what he does. But I bet he doesn't think we'll ever be as good as 
\emph{him.}"

"He's right," said Fred, feeling rather nervous. The Weasley twins did 
\emph{sometimes} disagree even when they had all the same information, but 
every time they did it seemed unnatural, like at least one of them must be 
doing something wrong. "This is \emph{Harry Potter} we're talking about. He can 
do the impossible. We can't."

"Yes we can," said George. "And we have to be \emph{more} impossible than him."

"But---" said Fred.

"It's what Godric Gryffindor would do," said George.

That settled it, and the twins snapped back into{\ldots} whatever it was that 
was normal for them.

"All right, then---"

"---let's think about it."