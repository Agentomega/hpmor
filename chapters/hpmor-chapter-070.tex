\partchapter{Self-Actualization}{V}

\lettrine{E}{ven} if you had been the Deputy Headmistress for three decades, and a Transfiguration Professor before that, it was rare that you saw Albus Dumbledore caught completely flatfooted.

“...Susan Bones, Lavender Brown, and Daphne Greengrass,” Minerva finished. “I should also note, Albus, that Miss Granger’s account of your seemingly unsupportive attitude—I believe her phrase was ‘he said I should be happy to be just a sidekick’—has generated a good deal of \emph{interest} among the older girls. Several of whom came to me to ask if Miss Granger’s accusations were true, since Miss Granger had said that I was there.”

The old wizard leaned back in his huge chair, still gazing at her, his eyes looking rather abstracted beneath the half-moon glasses.

“It placed me in something of a dilemma, Albus,” said Professor McGonagall. Her face stayed quite neutral, she made sure of that. “I now know that you did not truly mean to discourage the girl. Quite the opposite, in fact. But you and Severus have often told me that to keep a secret I must give no sign that differs from the reaction of someone truly ignorant. Thus I had no choice but to confirm that Miss Granger’s account was accurate, and feign the appropriate degree of worry, with a slight overtone of offense. After all, had I \emph{not} known you were deliberately manipulating Miss Granger, I might have been rather put out.”

“I...\ see,” the old wizard said slowly. His hands toyed absently with his silver beard, small quick gestures.

“Thankfully,” Professor McGonagall continued, “so far Professors Sinistra and Vector are the only two faculty members to don Miss Granger’s buttons.”

“Buttons?” repeated the old wizard.

Minerva drew forth a small silver disc bearing the initials \SPHEW, laid it on Albus’s desk, and gave it a brief tap with her finger.

And the voices of Hermione Granger, Padma Patil, Parvati Patil, Lavender Brown, Susan Bones, Hannah Abbott, Daphne Greengrass, and Tracey Davis cried out in unison, “\emph{We won’t settle for second best, it’s time to give a witch a quest!}”

“Miss Granger is selling them for two Sickles, and tells me that she has so far sold fifty of them. I believe that Nymphadora Tonks, in seventh-year Hufflepuff, is enchanting them for her. To conclude my report,” Professor McGonagall said briskly, “our eight newly minted heroines have asked permission to conduct a protest outside the entrance to your office.”

“I hope,” Albus said, frowning, “you explained to them that—”

“I explained to them that Wednesday at 7\PM would be fine,” said Minerva. She took back the button from the Headmaster’s desk, favored Albus with a honeyed smile, and turned to the door.

“Minerva?” said the old wizard from behind her. “\emph{Minerva!}”

The oaken door shut solidly behind her.
\sbreak
There wasn’t a lot of room between the brief stone walls that demarcated the vestibule to the Headmaster’s office, so although a lot of people had wanted to watch the protest, not many had been allowed to come. Just Professor Sinistra and Professor Vector, who were wearing the buttons, and the prefects Penelope Clearwater and Rose Brown and Jacqueline Preece, who were wearing the buttons. Behind \emph{them,} Professor McGonagall and Professor Sprout and Professor Flitwick, who weren’t wearing the buttons, scrutinizing the whole affair. Harry Potter and the Head Boy of Hogwarts were there, and the boy prefects Percy Weasley and Oliver Beatson, all wearing the buttons to show Solidarity. And of course the eight founding members of \SPHEW, forming a picket line next to the gargoyles with their signs. Hermione’s own sign, attached to a solid wooden handle which seemed to weigh heavier and heavier in her hands as the seconds passed, said \shout{Nobody’s Sidekick}.

And Professor Quirrell, who was leaning with his back against the far stone wall and watching with unreadable eyes. The Defense Professor had gotten one of her buttons, though she’d never sold one to him; and he wasn’t wearing it, but idly tossing it with one hand.

This whole idea had seemed like a much better idea four days ago, when the fires of her indignation had been burning fresh and hot, and she’d been facing the prospect of doing it all four days \emph{later} instead of \emph{right now.}

But she had to carry on, because that was what heroes did, they carried on, and also because it had seemed infinitely too awful to tell everyone she was calling it off. Hermione wondered how much heroism had gone on for reasons like that. Most books didn’t \emph{say} “And then they refused to give up, no matter how sensible it would have been, because that would’ve been too embarrassing”; but a great deal of history made a lot more sense that way.

At 7:15\PM, Professor McGonagall had told her, Headmaster Dumbledore would come down and talk to them for a couple of minutes. Professor McGonagall had said not to be frightened—the Headmaster was a good person deep down, and they’d properly gotten the school’s authorization for the protest.

But Hermione was very very aware that even if she was doing it with signed permission, she was still Defying Authority.

After she’d decided to be a hero, Hermione had done the obvious thing, and gone to the Hogwarts library and taken out books on how to be a hero. Then she’d returned those books back to their shelves, because it’d been patently obvious that none of the authors had been actual heroes themselves. Instead she’d just read five times over, until she’d memorized every word, the thirty inches by Godric Gryffindor that was all his autobiography and his life’s advice. (Or the English translation, anyway; she couldn’t read Latin yet.) Godric Gryffindor’s autobiography had been a lot more \emph{compressed} than the books Hermione was used to reading, he used \emph{one sentence} to say things that should’ve taken thirty inches just by themselves, and then there was \emph{another} sentence after that...

But it was clear from what she’d read that, while Defying Authority wasn’t the \emph{point} of being a hero, you couldn’t be a hero if you were too scared to do it. And Hermione Granger knew by now how others saw her, and she knew what other people thought she couldn’t do.

Hermione hefted her picket sign a little higher and concentrated on breathing slowly and rhythmically instead of hyperventilating until she fell over.

“\emph{Really?}” said Miss Preece in a tone of undisguised fascination. “They couldn’t \emph{vote?}”

“Indeed,” said Professor Sinistra. (The Astronomy Professor’s hair was still dark, and her dark face only slightly lined; Hermione \emph{would} have guessed her age at around seventy, except—) “I quite remember my mother’s rejoicing when they announced the Qualification of Women Act, although she did not actually qualify.” (Which meant that Professor Sinistra had been around her Muggle family in 1918.) “And that wasn’t the worst of it. Why, just a few centuries earlier—”

Thirty seconds later all the non-Muggleborns, male and female both, were staring at Professor Sinistra with utterly shocked expressions. Hannah had dropped her sign.

“And \emph{that} wasn’t the worst of it either, not by half,” finished Professor Sinistra. “But you see where this sort of thing could potentially lead.”

“Merlin preserve us,” said Penelope Clearwater in a strangled voice. “You mean \emph{that’s} how men would treat us if we didn’t have wands to defend ourselves?”

“\emph{Hey!}” said one of the boy prefects. “\emph{That’s} not—”

There was a short, sardonic laugh from the direction of Professor Quirrell. When Hermione turned her head to look she saw that the Defense Professor was still idly toying with the button, not bothering to glance up at the rest of them, as he said, “Such is human nature, Miss Clearwater. Rest assured that \emph{you} would be no kinder, if witches had wands and men lacked them.”

“I hardly think so!” snapped Professor Sinistra.

A cold chuckle. “I suspect it happens more often than any dare suggest, in the proudest pureblood families. Some lonely witch spies a handsome Muggle; and thinks how very easy it would be, to slip the man a love potion, and by him be adored alone and utterly. And since she knows he can offer her no resistance, why, it is only natural for her to take from him whatever she pleases—”

“\emph{Professor Quirrell!}” said Professor McGonagall sharply.

“I’m sorry,” Professor Quirrell said mildly, his eyes still looking down on the button in his hand, “are we all still pretending it doesn’t happen? My apologies, then.”

Professor Sinistra snapped, “And I suppose that wizards don’t—”

“There are \emph{children} present, Professors!” Again Professor McGonagall.

“Some do,” Professor Quirrell said equably, as though discussing the weather. “Although personally, I don’t.”

There was a bit of silence, for a time. Hermione put up her sign again—it had slipped down to her shoulder while she was listening. She’d never thought of that, not even a little, and now she was trying \emph{not} to think of it, and her stomach was feeling a bit queasy. She looked in Harry Potter’s direction, not quite knowing why she did; and she saw that Harry’s face was perfectly still. A chill ran down her spine before she looked away, not quite fast enough to miss the small nod that Harry gave her, as though they were agreeing on something.

“To be fair,” Professor Sinistra said after a while, “since I received my Hogwarts letter I can’t recall encountering any prejudice on account of being a woman, or colored. No, now it is all for being a Muggleborn. I believe Miss Granger said that it was \emph{just} with heroes that she found a problem, so far?”

It took Hermione a moment to recognize that she’d been asked the question, and then she said “Yes,” in a tone that squeaked a little. This whole thing had blown up a bit larger than she’d imagined when she’d started it.

“What exactly did you check, Miss Granger?” said Professor Vector. She looked older than Professor Sinistra, her hair starting to gray a little; Hermione hadn’t ever come close to Professor Vector in person until the Arithmancy Professor had asked her for a button.

“Um,” Hermione said, her voice a little high, “I checked the history books and there’s been as many woman Ministers of Magic as men. Then I looked at Supreme Mugwumps and there were a few more wizards than witches but not many. But if you look at people like famous Dark Wizard hunters, or people who’ve stopped invasions of Dark creatures, or people who’ve overthrown Dark Lords—”

“And the Dark Wizards themselves, of course,” said Professor Quirrell. \emph{Now} the Defense Professor had looked up. “You may add that to your list, Miss Granger. Among all the suspected Death Eaters we know of only two sorceresses, Bellatrix Black and Alecto Carrow. And I daresay that most wizards would be hard-pressed to name a single Dark Lady besides Baba Yaga.”

Hermione just stared at him.

He couldn’t \emph{possibly} be—

“Professor Quirrell,” said Professor Vector, “what exactly are you implying?”

The Defense Professor raised the button so that the golden-lettered \SPHEW faced them, and said, “Heroes,” then turned the button to show its silver backside and said, “Dark Wizards. They are similar career paths followed by similar people, and one can hardly ask why young witches are turning away from one course without considering its reflection.”

“Oh, \emph{now} I see!” said Tracey Davis, speaking up so suddenly that Hermione gave a small startle. “You’re joining our protest because you’re worried that not enough girls are becoming Dark Witches!” Then Tracey giggled, which Hermione couldn’t have managed at this point if you paid her a million pounds sterling.

There was a half-smile on Professor Quirrell’s face as he replied, “Not really, Miss Davis. In truth I do not care about that sort of thing in the slightest. But it is futile to count the witches among Ministers of Magic and other such ordinary folk leading ordinary existences, when Grindelwald and Dumbledore and He-Who-Must-Not-Be-Named were all men.” The Defense Professor’s fingers idly spun the button, turning it over and over. “Then again, only a very few folk ever do anything interesting with their lives. What does it matter to you if \emph{they} are mostly witches or mostly wizards, so long as \emph{you} are not among them? And I suspect you will not be among them, Miss Davis; for although you are ambitious, you have no ambition.”

“\emph{That’s not true!}” said Tracey indignantly. “And what’s it mean?”

Professor Quirrell straightened from where he had been leaning against the wall. “You were Sorted into Slytherin, Miss Davis, and I expect that you will grasp at any opportunity for advancement which falls into your hands. But there is no great ambition that you are driven to accomplish, and you will not \emph{make} your opportunities. At best you will grasp your way upward into Minister of Magic, or some other high position of unimportance, never breaking the bounds of your existence.”

Then Professor Quirrell’s gaze shifted away from Tracey, he was looking at \emph{her,} the pale blue eyes staring at her with an awful intensity\emph{—} “Tell me, Miss Granger. Do \emph{you} have an ambition?”

“Professor—” squeaked the high stern voice of Professor Flitwick, and then her Head of House’s voice cut off, and from the side of her vision Hermione saw that Harry had laid his hand on Professor Flitwick’s shoulder and was shaking his head, face looking very adult.

Hermione felt like a deer caught in headlights.

“What drove you to break your bounds, Miss Granger?” said the Defense Professor, still gazing directly at her. “Why is getting good marks in class no longer enough? Is it true greatness that you seek? Does some aspect of the world dissatisfy you, that you must remake according to your will? Or is this all merely a child’s game to you? I will be quite disappointed if this is only about rivaling Harry Potter.”

“I—” said Hermione, her voice so high-pitched it made a sort of peeping sound, but then she couldn’t think of what else to say.

“You may take a moment to think, if you like,” said Professor Quirrell. “Pretend it is a homework essay, six inches due Thursday. I hear you are quite eloquent in them.”

Everyone was looking at her.

“I—” said Hermione. “I don’t agree with one single thing you just said, anywhere.”

“Well spoken,” came Professor McGonagall’s crisp voice.

Professor Quirrell’s gaze did not waver. “That is not six inches, Miss Granger. \emph{Something} drives you to defy the Headmaster’s verdict and gather followers about yourself. Perhaps it is something you prefer not to speak aloud?”

Hermione knew the correct answer wouldn’t impress Professor Quirrell, but it was the correct answer, so she said it. “I don’t think you need ambition to be a hero,” Hermione said. Her voice wavered but it didn’t crack. “I think you just have to do what’s right. And they’re not my followers, we’re friends.”

Professor Quirrell leaned back against the wall again. The half-smile had faded from his face. “Most folk tell themselves they are doing right, Miss Granger. They do not thereby rise above the ordinary.”

Hermione took a couple of deep breaths, trying to be brave. “It’s not \emph{about} being not ordinary,” she said as stoutly as she could. “But I think if someone just tries to do what’s right, over and over again, and they’re not too lazy to do all the work it takes, and they think about what they’re doing, and they’re brave enough to do it even when they’re scared—” Hermione paused for an instant, her eyes darting to Tracey and Daphne, “—and they cleverly plan how to do it—and they don’t just do what other people do—then I think someone like that would already get into enough trouble.”

Some of the girls and boys chuckled, as did Professor McGonagall, who looked wry and proud at the same time.

“You may be right about that,” said the Defense Professor, his eyes half-lidded. He tossed Hermione the button, and she caught it without thinking. “My donation to your cause, Miss Granger. I understand that they are worth two Sickles.”

The Defense Professor turned and walked away without another word.

“I thought I was going to faint!” gasped Hannah after his footsteps had faded, and she heard some of the other girls letting out their breath or putting down their signs for a moment.

“I do \emph{too} have an ambition!” said Tracey, who seemed to be almost on the verge of tears. “I’m—I’m—I’ll figure out what it is by tomorrow, but I have one, I’m sure!”

“If you really can’t think of anything,” Daphne said, giving Tracey a comforting pat on the shoulder, “just go with the oldie but goodie and try to take over the world.”

“Hey!” said Susan sharply. “You’re supposed to be heroes now! That means you have to be \emph{good!}”

“No, it’s all right,” said Lavender, “I’m pretty sure General Chaos wants to take over the world and \emph{he’s} sort of a good guy.”

More conversation was going on behind the picket line. “My goodness,” said Penelope Clearwater. “I think that’s the most \emph{overtly} evil Defense Professor we’ve ever had.”

Professor McGonagall coughed warningly, and the Head Boy said, “You weren’t around for Professor Barney,” which made several people twitch.

“Professor Quirrell just \emph{talks} like that,” said Harry Potter, though he sounded less certain than before. “I mean, think about it, he doesn’t \emph{do} anything like what Professor Snape does—”

“Mr.~Potter,” squeaked Professor Flitwick, voice polite and face stern, “why did you ask me to stay silent?”

“Professor Quirrell was testing Hermione to see if he wanted to be her mysterious old wizard,” Harry said. “Which totally would not have worked out in any way, shape, or form, but she had to answer for herself.”

Hermione blinked.

Then Hermione blinked again, as she realized that it was Professor Quirrell who was Harry Potter’s mysterious old wizard, and not Dumbledore at all, and that \emph{really wasn’t a good sign—}

A rumbling noise filled the small stone vestibule, and Hermione, her nerves already on edge, spun rapidly around, almost dropping her protest sign as her other hand darted toward her wand.

The gargoyles were stepping aside, the Flowing Stone rumbling like rock as it moved like flesh. The huge ugly figures waited only briefly, dead gray eyes staring out in silent vigil. Then the great gargoyles folded their wings back into place and stepped back into their former positions, the Flowing Stone not changing its outward appearance at all as it returned from flexibility to motionlessness, and the brief gap in the stone of Hogwarts was solid once more.

And before them all, wearing robes of bright purple that probably only looked hideous if you were Muggleborn, stood the towering form of Albus Percival Wulfric Brian Dumbledore, the Headmaster of Hogwarts, the Chief Warlock of the Wizengamot, the Supreme Mugwump of the International Confederation of Wizards, the vanquisher of the Dark Lord Grindelwald and protector of Britain, the rediscoverer of the fabled Twelve Uses of Dragon’s Blood, the most powerful wizard alive; and he was looking at \emph{her,} Hermione Jean Granger, General of the recently expanded Sunshine Regiment, who was getting the best grades in the first year of Hogwarts classes, and who had declared herself a heroine.

Even her \emph{name} was shorter than his.

The Headmaster smiled benevolently at her, his wrinkle-lined eyes twinkling cheerfully beneath their half-circles of glass, and said, “Hello, Miss Granger.”

The odd thing was that it wasn’t nearly as scary as talking to Professor Quirrell. “Hello, Headmaster Dumbledore,” Hermione said with only a slight quaver in her voice.

“Miss Granger,” said Dumbledore, now looking more serious, “I think you and I may have had a bit of a misunderstanding. I did not mean to imply that you could not, or should not be a hero. I certainly did not mean to imply that witches in general should not be heroes. Only that you were...\ a bit young, to be thinking of such things.”

Hermione, unable to help herself, glanced at Professor McGonagall and saw that Professor McGonagall was giving her an encouraging smile—or she was giving the two of them \emph{some} kind of smile, anyway—so Hermione looked back at the Headmaster and said, the small quaver in her voice a little larger now, “Since you became Headmaster forty years ago, there’ve been eleven students to graduate Hogwarts who became heroes, I mean people like Lupe Cazaril and so on, and \emph{ten} of those were boys. Cimorene Linderwall was the only witch.”

“Hm,” said the Headmaster. There was a thoughtful expression on his face; he at least \emph{seemed} to be thinking about it. “Miss Granger, I have never been one for tallying such numbers. Often it is too much easier to count than to understand. Many good people have come out of Hogwarts, witches and wizards both; those famed as heroes are only one kind of good person, and perhaps not the highest. You did not include Alice Longbottom or Lily Potter in your reckoning... But leave that aside. Tell me, Miss Granger, did you tally how many heroes came out of Hogwarts in the forty years before me? For in that time I can recall only three now called heroes; and among those three, no witches at all.”

“I’m not trying to say it’s \emph{just} you!” Hermione said. “Only I think maybe a \emph{lot} of people, like the Headmasters before you too, maybe even your whole society and everything, might be discouraging girls.”

The old wizard sighed. His half-glasses eyes looked only at her, as though they were the only two people present. “Miss Granger, it might be possible to discourage witches from becoming Charms Mistresses, or Quidditch players, or even Aurors. But not heroes. If someone is meant to be a hero then a hero they will be. They will walk through fire and swim through ice. Dementors will not stop them, nor the deaths of friends, and not discouragement either.”

“Well,” Hermione said, and paused, struggling with the words. “Well, I mean...\ what if that’s not \emph{actually} true? I mean, to \emph{me} it seems that if you want more witches to be heroes, you ought to teach them heroing.”

“Many boys and girls are heroes in their dreams,” Dumbledore said quietly. He did not look at any of the other girls, only at her. “Fewer in the waking world. Many have stood their ground and faced the darkness when it comes for them. Fewer come for the darkness and force it to face them. It is a hard life, sometimes lonely, often short. I have told none to refuse that calling, but neither would I wish to increase their number.”

Hermione hesitated; there was something in the lined face that stopped her, like a hint to all the emotion that wasn’t being displayed, years and years of it...

\emph{Maybe if there were more heroes, their lives wouldn’t be so lonely, or so short.}

She couldn’t bring herself to say that, though, not to him.

“But the point is moot,” said the old wizard. He smiled, a bit ruefully she thought. “Miss Granger, you cannot teach heroism like you would teach Charms. You cannot assign twelve inches on how to carry on when all hope seems lost. You cannot rehearse students on when to stand up and tell the Headmaster he has done wrong. Heroes are born, not taught. And for whatever reason, more of them are born boys than girls.” The Headmaster shrugged, as if to say that \emph{he} was helpless to do anything about that.

“Um,” Hermione said. She couldn’t help it, she glanced behind her.

Professor Sinistra was looking a bit indignant. And it \emph{wasn’t} true that everyone was staring at her like she’d just been silly, the way she’d started to imagine while she was listening to Dumbledore.

Hermione turned back to face Dumbledore again, took a deep breath, and said, “Well, maybe people who are going to be heroes, will be heroes no matter what. But I don’t see how anyone could really \emph{know} that, aside from just saying it afterward. And when \emph{I} told you that I wanted to be a hero, you weren’t very encouraging.”

“Mr.~Potter,” the Headmaster said mildly. His eyes didn’t leave hers. “Please tell Miss Granger your impression of our own first meeting. Would you say that I was encouraging? Speak the truth.”

There was a pause.

“Mr.~Potter?” said Professor Vector’s voice from behind her, sounding puzzled.

“Um,” Harry’s voice said from further back, sounding extremely reluctant. “Um...\ well, actually in my case the Headmaster set fire to a chicken.”

“He \emph{what?}” Hermione blurted, only there were several other people exclaiming things at around the same time so she wasn’t sure anyone heard her.

Dumbledore went on gazing at her, looking perfectly serious.

“I didn’t know about Fawkes,” Harry’s voice said rapidly, “so he told me that Fawkes was a phoenix, while he was pointing to a chicken on Fawkes’s stand so I’d think \emph{that} was Fawkes, and then he set the chicken on fire—and also he gave me this big rock and told me it had belonged to my father and I ought to carry it everywhere—”

“But that’s \emph{crazy!}” Susan blurted out.

There was a sudden hush.

The Headmaster slowly turned his head to stare at Susan.

“I—” said Susan. “I mean—I—”

The Headmaster leaned down until he was face-to-face with the young girl.

“I didn’t—” said Susan.

Dumbledore put a finger to his lips and twiddled them, making a \emph{bweeble-bweeble-bweeble} sound.

“Albus,” said the weary voice of Professor McGonagall.

The Headmaster straightened up again and said, “Well, my good heroines, it has been pleasant speaking to you, but alas, much else remains to do this day. Still, rest assured that I am inscrutable at everyone, not just witches.”

The gargoyles stepped aside, the Flowing Stone rumbling like rock as it moved like flesh.

The huge ugly figures waited briefly with dead gray eyes staring out in silent vigil, as Albus Percival Wulfric Brian Dumbledore, smiling as benevolently as when he’d first emerged from his office, stepped back into the Enchantment of the Endless Stair.

Then the great gargoyles folded their wings back into place and stepped back into their former positions, only one last brief “Bwa-ha-ha!” echoing out before the gap closed.

There was a long silence.

“He \emph{really} set a chicken on fire?” said Hannah.
\sbreak
The eight of them had continued protesting even after that, but to be honest their heart had gone out of it.

It \emph{had} been established, after some careful questions from Professor Flitwick, that Harry Potter hadn’t smelled the chicken burning. Which meant that it had probably been a pebble or something, Transfigured into a chicken and then enclosed in a Boundary Charm to make sure that no smoke escaped into the air—both Professor Flitwick and Professor McGonagall had been very emphatic about nobody trying that without their supervision.

But still...

But still...\ what?

Hermione didn’t even \emph{know} but still what.

But \emph{still}.

After a lot of glances exchanged between girls none of whom had wanted to be first to say it, Hermione had declared the protest over, and the adults and boys had drifted off.

“You don’t think we were being unfair to Dumbledore, do you?” said Susan as the heroines walked away to the sound of eight pairs of feet trodding on the stone paving of Hogwarts’s corridors. “I mean, if he \emph{is} crazy at everyone and not just at witches then it’s not discrimination, right?”

“I don’t want to protest against the Headmaster any more,” Hannah said weakly. The Hufflepuff girl seemed a bit unsteady on her feet. “I don’t care what Professor McGonagall says about him not holding it against us, it’s just too much for my nerves.”

Lavender snorted. “I guess \emph{you} won’t be slaying armies of Inferi anytime soon—”

“Stop that!” Hermione said sharply. “Look, all of us have got to \emph{learn} to be heroines, right? It’s okay if someone doesn’t know right away.”

“The Headmaster doesn’t think it \emph{can} be learned,” Padma said. The Ravenclaw girl’s face was thoughtful, her steps measured as she strode through the corridor. “The Headmaster doesn’t even think that’s a good idea.”

Daphne was striding with her back straight and her head held bolt upright, looking more like a Proper Young Lady in her Hogwarts robes than Hermione could have done with her best formal dress. “The Headmaster,” Daphne said in a precise voice, her shoes making hard, sharp tacking sounds on the stone, “thinks the lot of us are a bunch of silly girls playing games, and that someday Hermione might make a good sidekick but the rest of us are hopeless.”

“Is he \emph{right?}” said Parvati. The Gryffindor girl’s face was very serious, making her look much more like her twin than she usually did. “I mean it has to be asked—”

“\emph{No!}” spat Tracey. The Slytherin girl was stalking through the hallway looking ready to \emph{kill} someone, like a miniature female Snape. Of all the girls, Tracey was the one who Hermione knew least. Hermione had talked to Lavender once before, but she’d never really \emph{seen} Tracey except at wandpoint during a battle, until the Slytherin had jumped up from her sofa to volunteer. Right now Tracey looked so angry there should’ve been sparks flying off her. “We’ll show him! We’ll show them \emph{all!}”

“Okay,” said Susan, “that was \emph{definitely} evil—”

“No,” said Lavender, “that’s a Chaos Legion motto, actually. Only she didn’t do the insane laughter.”

“That’s right,” Tracey said, her voice low and grim. “This time I’m not laughing.” The girl went on stalking through the corridor, like she had dramatic music accompanying her that only she could hear.

(Hermione was starting to worry about what \emph{exactly} the impressionable youths of the Chaos Legion were learning from Harry Potter.)

“But—I mean—” Parvati said. She still had a contemplative look on her face. “I mean, you can see \emph{why} the Headmaster would think we were just silly girls, right? What does protesting outside the Headmaster’s office have to do with becoming heroines?”

“Huh,” Lavender said, now looking thoughtful herself. “That’s true. We should do something heroic. I mean heroinic.”

“Um—” said Hannah, which very much expressed Hermione’s own feelings on the subject.

“Well,” said Parvati, “has everyone already been through Dumbledore’s third-floor forbidden corridor? I mean everyone in Gryffindor’s been through it by now—”

“Hold \emph{on!}” Hermione said desperately. “I don’t want you doing anything \emph{dangerous!}”

There was a pause while everyone looked at Hermione, who was suddenly realizing, much too late, why Dumbledore hadn’t wanted anyone \emph{else} to be a hero.

“I don’t think you can become a heroine if you never do anything dangerous,” Lavender observed reasonably.

“Besides,” said Padma, a considering look on her face. “Everyone knows that nothing \emph{really} bad ever happens in Hogwarts, right? To students, I mean, not to the Defense Professors. We’ve got all these ancient wards and so on.”

“Um—” Hannah said again.

“Yeah,” said Parvati, “the worst that can happen is that we’ll lose a few dozen House points or something, and there’s two of us from each House so \emph{that’ll} all come out even.”

“Why, that’s \emph{brilliant,} Hermione!” said Daphne in a tone of great amazement. “The way you set it up means we can get away with \emph{anything!} And I didn’t even notice your cunning plan until now!”

“\emph{UM}—” said Hermione, Hannah, and Susan.

“Right!” said Parvati. “So now it’s time for us to become real heroines. We’ll come for the darkness—”

“And make \emph{it} face \emph{us} —” said Lavender.

“And teach it to be afraid,” Tracey Davis said grimly.
